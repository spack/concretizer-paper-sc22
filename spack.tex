\spack is a package manager designed to support High Performance Computing (HPC) use-cases.
\spack allows multiple installations of a single package to coexist on the system to support scientific research and the requirements of scientific software
This greatly expands the space of possible build configurations necessary to support, and is the major driver behind other design decisions in \spack.

At a high level \spack can be broken down into three primary components: A ``spec syntax'' to represent individual builds from the combinatorial build space, a ``package DSL'' to specify build instructions for individual packages, and the ``concretizer'', the internal algorithm that takes abstract requests from the user as input and outputs concrete decisions about how to build a package or set of packages.
While this paper is primarily focused ont he concretizer, we will discuss all three components in this section for background.

\subsection{Spec Syntax}

The combinatorial space of possible \spack builds is huge.
For a given package, decisions must be made about what version to build, what options to build with, what architecture to target, and what compiler to use.
Many packages have tens of direct and transitive dependencies, and for each dependency the same decisions must be made.
Furthermore, there are packages that are interchangeable at the Application Programming Interface (API) level, which can be chosen among arbitrarily for a particular build -- the cannonical example in the HPC community being the Message Passing Interface (MPI) implementations.
We will refer to such interchangeable components as ``virtual packages'' and ``virtual dependencies''.
For each virtual dependency, even more options arise to complicate the build space as an implementation must be selected in addition to all of the decisions necessasary to build that implementation.

The spec syntax in \spack allows users to specify points or spaces within this combinatorial build space.
For example, often a users may only care to build a particular package, with no additional constraints.
This portion of the build space is all possible builds of that package, but not other packages.
It can be represented in the spec syntax by the package name.
\mint{console}|hdf5  # specifies the space of any build of the package hdf5|

Particular sigils are used to specify other attributes of the package that the user may care about. This allows the user to control the version, compiler, compiler version, architecture, and build options for the package.
\begin{tabular}{cll}
\hline
Sigil & Examples & Meaning \\
\hline
\texttt{\%} & \texttt{hdf5\%gcc} & Constrain the build space to use a particular compiler. \\
\hline
\texttt{@} & \shortstack{\texttt{hdf5@1.10.2} \\ \texttt{hdf5\%gcc@10.3.1}} & \shortstack{Constrain the build space to a particular version or set of versions \\ Constrain the build space to a particular compiler version or set of compiler versions} \\
\hline
\shortstack{\texttt{+} \\ \texttt{~} \\ \texttt{-}} & \shortstack{\texttt{hdf5+mpi} \\ \texttt{hdf5~mpi} \\ \texttt{hdf5 -mpi}} & Constrain the build space to require (\texttt{+}) or forbid (\texttt{~, -}) a boolean build option. \\
\hline
\texttt{key=value} & \shortstack{\texttt{hdf5 mpi=true} \\ \texttt{hdf5 api=default} \\ \texttt{hdf5 target=skylake}} & Constrain the build space to require a particular build option value or architecture value. \\
\end{tabular}


The spec syntax is telescoping and fully recursive.
Recursion is what allows full control over the build space -- any attribute that can be set on the root can be set on any dependency. For example, the spec below constrains the build to any possible build of the package \texttt{hdf5} at version 1.10.2 that depends on a build of the package \texttt{zlib} built with the \texttt{gcc} compiler and a build of the package \texttt{cmake} built to target the \texttt{aarch64} architecture.

\begin{verbatim}
hdf5@1.10.2 ^zlib%gcc ^cmake target=aarch64
\end{verbatim}

While in theory it is possible to completely constrain a build via the spec syntax, we find that in practice users care about a very small set of constraints, and rely on the concretizer to fill in defaults for the rest of the build. We refer to a portion of the build space in \spack as a ``spec'', and a single point as a ``concrete spec''. A spec is represented internally as a directed acyclic graph (DAG) where edges are dependency relationships and nodes are packages.

\subsection{Package DSL}

The package DSL in \spack is what allows developers to expose such a huge combinatorial space of possible builds.
A \spack package consists of two parts: metadata on the requirements for any build of the package, and build instructions.

Unlike many systems, a package in \spack is fully templated.
The input to the build instructions includes a fully concrete spec, and the build instructions translate that spec into commands to the build system of the package.

The package metadata requires more explanation.
The package metadata defines the parameters of the build space for the package.
This is where the package defines the build options, called ``variants'', that can be referenced for that package by the spec syntax.
It is also where versions of the package are listed.
The package metadata also constraints the build space for the package. For example, a package might register conflicts with a particular compiler with which it is known not to compile.
Lastly, and most interestingly, the package metadata defines the dependency relationships among packages.
The package may depend on any spec for the dependency using the spec syntax discussed above.

Consider this example package:

\begin{verbatim}
from pkgkit import *  # import the DSL

class Foo(Package):  # this is the classname for the package `foo`

    version("1.1.0")
    version("1.0.0")

    variant("bar", default=True, description="Build foo with bar support")

    depends_on("bar@2.0:", when="+bar")  # depends on bar 2.0 or later for bar support
    depends_on("baz")  # depends on baz
    depends_on("baz+corge", when="@1.1.0:")  # newer versions require baz with corge support

    conflicts("%intel")  # known failure when building with intel compilers
    conflicts("target=aarch64:)  # does not support architectures derived from ARM64

    def install(self, spec, prefix):
        # translate spec into arguments to build system
        config_args = [
            'baz=%s' % spec['baz'].prefix
        ]

        if 'bar' in spec:
            config_args.append('bar=%s' % spec['baz'].prefix)
        else:
            config_args.append('bar=none')

        # run the build system -- configure and make methods automatically generated
        configure(*config_args)
        make()
        make('install')
\end{verbatim}

This is a simple package.
It has two versions, 1.0.0 and 1.1.0.
It has one build option, controlling some functionality and requiring an additional dependency.
Its dependency requirements vary slightly by the version of the package.
It has some configurations in which it's known not to build successfully.
That information is all given to the concretizer, and once a concrete spec is generated, the build instructions are fairly simple.

\subsection{Concretization}

There are three primary inputs to concretization in \spack, and we have already discussed two of them.
The inputs to the concretization algorithm are the package constraints, the abstract spec input by the user, and the user configuration.

The user configuration allows the user to change the default behaviors of \spack, but is not necessary to understand the workings of the concretizer.
In the following discussion of the concretizer, in many places the algorithm will ``choose the default'' or ``choose the most preferred ...'' when otherwise unconstrained.
In all of those cases, \spack has builtin defaults that can be overriden with user configuration.

On a conceptual level, we can think of the concretizer as accomplishing two tasks.
It determines the set of packages needed in the spec DAG, and it fills in any missing information about how to build each node in the DAG.
For each decision it makes, it first considers the constraints of the build space generated by the package, constrains those further by the abstract spec, and chooses any additional degrees of freedom from the preferences expressed in the user configuration.

Consider our package \texttt{foo} from the  ``Package DSL'' section, and we will concretize the abstract spec \texttt{foo@1.0.0 \^{}baz@3.1.1}.
We will assume for the moment that neither \texttt{bar} nor \texttt{baz} have any additional dependencies.
Then the set of packages and constraints needed for this abstract spec is

\begin{itemize}
\item \texttt{foo@1.0.0} constraint from the abstract spec \\
\item \texttt{bar@2.0:} constraint from foo package file \\
\item \texttt{baz@3.1.1} constraint from abstract spec \\
\end{itemize}

None of these nodes are fully concrete, so additional decisions must be made from preferences.
We will consider only \texttt{foo} since we have its package file, but the same process would apply to \texttt{bar} and \texttt{baz}.

For \texttt{foo}, the version is already fully constrained.
Barring user configuration, the default compiler in \spack is the newest version of \texttt{gcc} available on the system, so the concretizer may choose \texttt{gcc@11.2.0}.
For architecture options, the default is the current architecture. We will assume the algorithm is running on a linux system running \texttt{centos8} on a \texttt{skylake} CPU.
Our package file for \texttt{foo} specifies the default value for the variant \texttt{bar} is True, so the concretizer chooses that value in the absence of user configuration to the contrary.
This creates a feedback loop with the list of nodes, increasing the complexity of the algorithm.
The concrete spec generated would be

\begin{verbatim}
foo@1.0.0+bar%gcc@11.2.0 arch=linux-centos8-skylake
  ^ bar@2.0.0 ...
  ^ baz@3.1.1 ...
\end{verbatim}

where the elided values are the details of the dependency packages we did not consider for this exercise.
