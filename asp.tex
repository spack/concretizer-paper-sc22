%-------------------------------------------------------------------------------
\section{Answer Set Programming}
%-------------------------------------------------------------------------------
\label{sec:asp}

The concretization problem is essentially a combinatorial search problem. While many
package managers use homegrown SAT solvers for this type of problem, encoding the
complexity of our domain in pure boolean SAT would be extrmely tedious. To manage the
complexity, we leverage Answer Set Programming (ASP), a form of declarative programming
that allows us to formally specify our semantics in first-order logic with variables and
quantification, in addition to boolean operators. The input language for ASP is similar
to Prolog~\cite{baral_2003}. Unlike Prolog, ASP has no operational semantics and is not
Turing-complete. Rather, ASP {\it grounds} a first-order logic program to a {\it
  propositional} (no variables) logic, and reduces the result to SAT with optimization.
Also unlike Prolog, ASP programs are guaranteed to terminate.

In the remainder of this section, we illustrate how ASP can simplify the specification
and maintenance of our concretization algorithm while also affording strong guarantees
than we can implement ourselves with simple heuristics.

\subsection{ASP Syntax}

ASP programs are comprised of ``terms'', which can be boolean atoms or a functions whose
arguments may also be terms. A term followed by a period ({\tt .}) is called a ``fact''.
The following is a simple program comprised entirely of facts:

\begin{minted}[fontsize=\small, bgcolor=bg]{prolog}
  optimize_for_reuse.
  node("hdf5").
  depends_on("hdf5", "mpi").
\end{minted}

Note that these functions are not imperative; rather this snippet can be ready roughly
as ``{\tt optimize\_for\_reuse} is enabled. A node called {\tt hdf5} exists. {\tt hdf5} depends on {\tt mpi} ''.

In addition to facts, ASP programs contain rules, which can derive additional facts. An
ASP rule has a {\it head} and a {\it body}, separated by \texttt{:-}. The \texttt{:-}
can be read as ``if'' --- the head (left side) is true if the body (right side) is true.
Terms in the body of a rule can also be preceded by the keyword ``not'' to imply the
head based on their negation. Logical ``and'' is represented by a comma in the rule
body, and ``or'' is represented by repeating the head with a different rule body.

ASP progrmas can contain variables, represented by capitalized words. Variables are
scoped to the rule or fact in which they appear; the variable {\tt Package} may be used
in unrelated rules without any scoping issues Rules are instantiated with all possible
substitutions for variables. For example:

%Additionally, variables appearing in the
%head of a rule must appear in a positive (non-negated) term in the rule body. This is
%referred to as ``safety'' in ASP, and is essential to ensuring the algorithm will
%terminate.

\begin{minted}[fontsize=\small, bgcolor=bg]{prolog}
  optimize_for_reuse.
  node("hdf5").
  depends_on("hdf5", "mpi").
  node(Dependency) :- node(Package),
                      depends_on(Package, Dependency).
\end{minted}

The rule here translates roughly to ``If a package is in the graph and it depends on
another package, that package must be in the graph'', and the rule in the program above
will derive the fact {\tt node("mpi")} when it is instantiated with {\tt
  node("hdf5")} and {\tt depends\_on("hdf5", "mpi")}. This is the essenence of how
  we model dependencies in ASP.

Two additional types of rules are important to understand the power of ASP:
\textit{integrity constraints} and \textit{choice rules}.
%Answer Set Programming gets its name from the set of possible answers which satisfy the constraints of the program.
%ASP solvers focus on understanding the constraints on this space.
Integrity constraints allow the programmer to rule out swathes of the answer set space
by specifying conditions not to allow. In ASP, these are represented by a headless rule.

\begin{minted}[fontsize=\small, bgcolor=bg]{prolog}
  optimize_for_reuse.
  node("hdf5").
  depends_on("hdf5", "mpi").
  node(Dependency) :- node(Package),
                      depends_on(Package, Dependency).
  :- depends_on(Package, Package).
\end{minted}

Here our integrity constraint says ``A package cannot depend on itself''.
Choice rules give the solver freedom to choose from possible options. A choice rule is
(optionally) constrained on either side to specify the minimum (left) and maximum
(right) number of choices the solver can select.

\begin{minted}[fontsize=\small, bgcolor=bg]{prolog}
  optimize_for_reuse.
  node("hdf5").
  depends_on("hdf5", "mpi").
  node(Dependency) :- node(Package),
                      depends_on(Package, Dependency).
  :- depends_on(Package, Package).

  possible_version("hdf5", "1.13.1").
  possible_version("hdf5", "1.12.1").
  1 { version(P, V) : possible_version(P, V) } 1
    :- node(P).
\end{minted}

The choice rule on the last line means ``If a package is in the graph, assign it
exactly one of its possible versions.'' The $\{a:b\}$ syntax here has the same meaning
as in formal set theory; it denotes the set of all $a$ such that $b$ is true. The
numerals on either side of the choice rule are {\it cardinality constraints}; they specify
lower and upper bounds on the size of the set. Choice rules are similar to guessing;
they give the solver options for paths to explore next.

%allow the solver substantial freedom to create facts, within the specified
%constraints, that satisfy other constraint of the program or maximize optimization
%criteria.

In addition to constraints on the solution space, ASP allows optimization criteria.
Optimization criteria allow the solver to return a single answer set from the among all
valid answer sets, and guarantee that the chosen answer set minimizes or maximizes one
or many criteria. There are often many valid solutions to a dependency resolution query,
but only the optimal solution is relevant once we find it.

\begin{minted}[fontsize=\small, bgcolor=bg]{prolog}
  optimize_for_reuse.
  node("hdf5").
  depends_on("hdf5", "mpi").
  node(Dependency) :- node(Package),
                      depends_on(Package, Dependency).
  :- depends_on(Package, Package).

  possible_version("hdf5", "1.13.1", 0).
  possible_version("hdf5", "1.12.1", 1).

  1 { version(P, V) : possible_version(P, V) } 1
    :- node(P).

  version_weight(P, V, Weight)
    :- possible_version(P, V, Weight).
  #minimize{ W@3,P,V : version_weight(P, V, W)}.
\end{minted}

This optimization constraint says ``I prefer (at priority 3) solutions that minimize the
sum of the weights (W) of the package (P) versions (V), for all packages and versions.''
A detailed discussion of the syntax for optimization is beyond the scope of this paper;
it suffices to know that multi-objective optimization across a variety of criteria is
possible in ASP.

Our small and very incomplete program here already shows the power of ASP. In a SAT
solver, the programmer would need to write out explicitly the rule for deriving
dependency nodes from dependent nodes for every possible pair of packages. Even worse,
our choice rule for selecting versions would need to be written out for every
combination of package and version, and an enormous combination of logical operators
constructed to ensure that for any group of versions \texttt{$a_1, ..., a_n$}, one of
the constructions \texttt{$a_i$} and not any \texttt{$a_j$} for all $i\neq{j}$. ASP
makes this much simpler.

%Writing
%out the facts of our small program in SAT is hard enough, and we haven't even discussed
%implementing optimization criteria in boolean logic. In ASP, though, it appears
%reasonably straightforward and is legible to future maintainers of our code.

\subsection{ASP Solvers}

ASP solvers search the input program for {\it stable models}, also known as
\textit{answer sets}. An answer set is a set of true atoms for which every rule in the
input program is idempotent. It is similar to a fixed point, or the solution to a system
of (boolean) equations.
%
%%ASP solvers utilize a huge variety of
%methods to find answer sets. We will focus on the structure of the \texttt{clingo}
%solver that we use in \spack.
%
%\paragraph{Clingo}
%
In this paper, we use the popular \textit{clingo}~\cite{gebser+:aicomm11} tool. which
consists of two components: the grounder (\textit{gringo}) and the solver
(\textit{clasp}). \texttt{clasp} uses a CDCL SAT-like algorithm and can also do
MaxSAT-style optimizations. \texttt{gringo} ``grounds'' the first-order input problem to
produce the propositional form accepted by {\tt clasp}.
%
While the internals of \textit{clasp} and \textit{gringo} are outside the scope of this
work, it is important to understand that it guarantees completeness and optimality.
There are no set of inputs for which \textit{clingo} will return a false negative (claim
the rules are incompatible if they are not) and solutions returned by
\textit{clingo} are guaranteed to be optimal.

%% TODO: This could probably use some work here
