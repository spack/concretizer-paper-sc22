%-------------------------------------------------------------------------------
\section{Conclusions}
%-------------------------------------------------------------------------------
\label{sec:conclusions}

The complexity of software dependencies and optimization needs in HPC creates a
particular challenge for package management in the HPC space. Ad-hoc techniques for
dependency resolution have proven to require substantial investment of programmer hours
to manage even a small subset of the possible space of HPC software configurations.

In this paper we introduced a new dependency resolution method for \spack{}, an HPC
package manager. This new dependency resolution method uses Answer Set Programming to
model Spack's DSL, compatibility semantics, and optimization rules in a maintainable,
declarative syntax. It has allowed us to implement new functionality for \spack{}, such
as targeted reuse of installations and binary packages, that was simply infeasible with
previous methods reliant on a greedy fixed-point algorithm.

We showed that the performance of the new concretizer is competitive in most cases with
the previous algorithm, and that performance of the reuse capablity is scalable.

%The improvements we introduced in this paper are not merely theoretical --- they are seeing worldwide use in production through the most recent major version release of \spack{}.
