The \clingo{} solver performance given a logic program depends
on a number of factors. First, the number of facts in a specific concretization. Second,
the configuration and various optimization parameters passed to the solver. 

The solving process consists of four stages: \emph{setup}, \emph{load}, 
\emph{ground}, and \emph{solve}. The first two are preliminary phases and the other two 
perform the actual solve. Specifically, the setup phase generates the facts for the given
spec, whereas the load phase loads the main logic program (i.e., the rules of the software 
model) as a resource into the solver. The grounding phase is the first part of the actual
solve, it turns the facts and first-order logic rules into propositional logic. Once we
have a grounded program, we can run the last phase, which is the full solve in \clingo{}.

We instrumented the solving code such that we obtained time measurements for each one
of the phases and the total time for the whole process.


\subsection{Solve timings for all packages}


\begin{figure*}[htb]

    \centering
    \subfloat[][Ground times]{
    \label{subfig:cdf_quartz_ground}
        \includegraphics[width=\perfsubfigwidth\textwidth]{figures/perf/cdf_quartz_ground_fig.pdf}
    }\hfill%
    \subfloat[][Solve times]{
    \label{subfig:cdf_quartz_solve}
        \includegraphics[width=\perfsubfigwidth\textwidth]{figures/perf/cdf_quartz_solve_fig.pdf}
    }\hfill%
    \subfloat[][Full solving times]{
    \label{subfig:cdf_quartz_full}
        \includegraphics[width=\perfsubfigwidth\textwidth]{figures/perf/cdf_quartz_total_fig.pdf}
    }
    \caption{Cumulative distribution of solve times across all packages on the Quartz machine.}
    \label{fig:cdf_quartz}

\end{figure*}



\begin{figure*}[htb]

    \centering
    \subfloat[][Ground times]{
    \label{subfig:cdf_lassen_ground}
        \includegraphics[width=0.32\textwidth]{figures/perf/cdf_lassen_ground_fig.pdf}
    }\hfill%
    \subfloat[][Solve times]{
    \label{subfig:cdf_lassen_solve}
        \includegraphics[width=0.32\textwidth]{figures/perf/cdf_lassen_solve_fig.pdf}
    }\hfill%
    \subfloat[][Full solving times]{
    \label{subfig:cdf_lassen_full}
        \includegraphics[width=0.32\textwidth]{figures/perf/cdf_lassen_total_fig.pdf}
    }
    \caption{Cumulative distribution of solve times across all packages on the Lassen machine.}
    \label{fig:cdf_lassen}

\end{figure*}



\begin{figure*}[htb]

    \centering
    \subfloat[][Ground times]{
    \label{subfig:deps_quartz_load}
        \includegraphics[width=\perfsubfigwidth\textwidth]{figures/perf/deps_quartz_ground_fig.pdf}
    }\hfill%
    \subfloat[][Solve times]{
    \label{subfig:deps_quartz_solve}
        \includegraphics[width=\perfsubfigwidth\textwidth]{figures/perf/deps_quartz_solve_fig.pdf}
    }\hfill%
    \subfloat[][Full solving times]{
    \label{subfig:deps_quartz_full}
        \includegraphics[width=\perfsubfigwidth\textwidth]{figures/perf/deps_quartz_total_fig.pdf}
    }
    \caption{Solve times vs. number of dependent packages across all packages on the Quartz machine.}
    \label{fig:deps_quartz}

\end{figure*}



\begin{figure*}[htb]

    \centering
    \subfloat[][Load times]{
    \label{subfig:deps_lassen_load}
        \includegraphics[width=0.32\textwidth]{figures/perf/deps_lassen_load_fig.pdf}
    }\hfill%
    \subfloat[][Solve times]{
    \label{subfig:deps_lassen_solve}
        \includegraphics[width=0.32\textwidth]{figures/perf/deps_lassen_solve_fig.pdf}
    }\hfill%
    \subfloat[][Full solving times]{
    \label{subfig:deps_lassen_full}
        \includegraphics[width=0.32\textwidth]{figures/perf/deps_lassen_total_fig.pdf}
    }
    \caption{Solve times vs. number of dependent packages across all packages on the Lassen machine.}
    \label{fig:deps_quartz}

\end{figure*}



\begin{figure*}[htb]

    \centering
    \subfloat[][Setup times]{
    \label{subfig:cdf_e4s_quartz_load}
        \includegraphics[width=0.32\textwidth]{figures/perf/cdf_e4s_cache_quartz_setup_fig.pdf}
    }\hfill%
    \subfloat[][Solve times]{
    \label{subfig:cdf_e4s_quartz_solve}
        \includegraphics[width=0.32\textwidth]{figures/perf/cdf_e4s_cache_quartz_solve_fig.pdf}
    }\hfill%
    \subfloat[][Total solver times]{
    \label{subfig:cdf_e4s_quartz_full}
        \includegraphics[width=0.32\textwidth]{figures/perf/cdf_e4s_cache_quartz_total_fig.pdf}
    }
    \caption{Cumulative distribution of solve times for different cache sizes across all E4S packages on the Quartz machine.}
    \label{fig:cdf_e4s_quartz}

\end{figure*}



\begin{figure*}[htb]

    \centering
    \subfloat[][Setup times]{
    \label{subfig:cdf_e4s_lassen_load}
        \includegraphics[width=0.32\textwidth]{figures/perf/cdf_e4s_cache_lassen_setup_fig.pdf}
    }\hfill%
    \subfloat[][Solve times]{
    \label{subfig:cdf_e4s_lassen_solve}
        \includegraphics[width=0.32\textwidth]{figures/perf/cdf_e4s_cache_lassen_solve_fig.pdf}
    }\hfill%
    \subfloat[][Total solver times]{
    \label{subfig:cdf_e4s_lassen_full}
        \includegraphics[width=0.32\textwidth]{figures/perf/cdf_e4s_cache_lassen_total_fig.pdf}
    }
    \caption{Cumulative distribution of solve times for different cache sizes across all E4S packages on the Lassen machine.}
    \label{fig:cdf_e4s_quartz}

\end{figure*}


- Time vs. number of dependnecies
- other properties?
- --single-shot vs. no single-shot?
- different tactics?

\subsection{Solve timings for all packages with reuse}
- scaling repo size


How is the ASP based solver performing?
