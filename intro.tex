%-------------------------------------------------------------------------------
\section{Introduction}
%-------------------------------------------------------------------------------
\label{sec:intro}

Managing dependencies for scientific software has been notoriously difficult for
years~\cite{hoste+:pyhpc12,gamblin+:sc15,dubois2003johnny,hochstein+:2011-build}.
Scientific software is typically built from source to achieve the best performance, and
configuring build systems to target speicfic machines, dependency versions, and
compilers requires painstaking care if done manually. In the early 2010's, two main
package managers emerged to tackle this problem for the HPC space:
Spack~\cite{gamblin+:sc15} and EasyBuild~\cite{hoste+:pyhpc12}.


\cite{}

E4S is ~100 packages and 400 dependencies
many conditional dependencies
many version constraints
many optionally enabled packages
many virtual packages






Our contributions are:

\begin{enumerate}
\item A general mapping of HPC dependency compatibility semantics and optimization rules to ASP;
\item A technique to optimize for reuse of existing builds in combinatorial package solves;
\item An implementation of ASP solver in {\it Spack}, a popular HPC package manager; and
\item An evaluation of our system's performance on large, real-world HPC package repositories.
\end{enumerate}
