At a high-level, with the introduction of \clingo, each concretization
combines:
\begin{enumerate}
\item A large number of \emph{facts} characterizing the specific problem being solved
\item A small \emph{logic program} encoding the rules and constraints of the software model
\end{enumerate}
The facts are always generated starting from one or more 
\emph{root specs} that need to be concretized, and they account 
for both the metadata contained in package recipes and the 
current state of \spack{} in terms of configuration and 
installed software.
For instance, a simple directive such as:
\begin{minted}[fontsize=\small, bgcolor=bg]{python}
version('1.2.11', sha256='ah45rstgef...')
\end{minted}
in \texttt{zlib}'s recipe is translated to the following fact:
\begin{minted}[fontsize=\small, bgcolor=bg]{text}
version_declared("zlib", "1.2.11", 0, "package_py").
\end{minted}
in our ASP solve. Similarly, the request to concretize
\begin{minted}[fontsize=\small, bgcolor=bg]{text}
zlib@1.2.11
\end{minted}
generates the following three facts:
\begin{minted}[fontsize=\small, bgcolor=bg]{text}
root("zlib").
node("zlib").
version_satisfies("zlib","1.2.11").
\end{minted}
stating that \texttt{zlib} is a root node and should satisfy
a version requirement.
A typical solve has around $10k-100k$ facts which include
the encoding of dependencies, variants, preferences, etc.

The logic program encodes the software model used by \spack{}
and is solely composed by first-order rules, integrity 
constraints and optimization objectives. 
The declarative nature of ASP makes it easy to enforce 
certain properties on the optimal solution. For instance, 
these three lines ensure that we never have a cyclic dependency
in a DAG:
\begin{minted}[fontsize=\small, bgcolor=bg]{text}
path(A, B) :- depends_on(A, B).
path(A, C) :- path(A, B), depends_on(B, C).
:- path(A, B), path(B, A).
\end{minted}
The first rule establish the base case: if \texttt{A} depends
on \texttt{B} there is a path from \texttt{A} to \texttt{B}.
The second rule defines paths to transitive dependencies 
with a recursive definition: if there is a path from \texttt{A} 
to \texttt{B} and \texttt{B} depends on \texttt{C}, there is
a path from \texttt{A} to \texttt{C}. The final line is an 
\emph{integrity constraints} stating that paths from \texttt{A} 
to \texttt{B} and paths from \texttt{B} to \texttt{A} cannot 
occur at the same time in a valid solution.

To give a rough idea of the compactness and expressiveness of 
the ASP encoding, the entire logic program for the software 
model described in Section~\ref{sec:software-model} is around 
$500$ lines of code. From the point of view of the application, 
the concretization process is also straightforward to follow
conceptually:
\begin{enumerate}
\item Facts are generated for the current concretization
\item The logic program and the facts are sent to the solver
\item The concretized DAG is rebuilt from the result
\end{enumerate}
Finally, it's worth stressing that the logic program changes only
when the underlying software model changes, as opposed 
to the generated facts that are different whenever the root
spec to be concretized or \spack's configuration changes.

\subsection{Optimization Objectives}

\subsection{Generalized Condition Handling}
  - examples of: conflicts, virtuals, dependencies, variants

\subsection{Example: Specialize LAPACK Providers}
\subsection{Example: CUDA/RoCM Conditional Variants}
