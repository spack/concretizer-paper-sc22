%-------------------------------------------------------------------------------
\section{Modeling Software Dependencies with ASP}
%-------------------------------------------------------------------------------
\label{sec:asp-model}

At a high-level, using \clingo, our concretizer combines:
\begin{enumerate}
\item Many \emph{facts} characterizing the problem instance;
\item A small \emph{logic program} encoding the rules and constraints of the software model; and
\item Optimization criteria that define an {\it optimal} model.
\end{enumerate}
The facts are always generated starting from one or more \emph{root specs}, and they
account for metadata from all package recipes of possible dependencies, as well as
the current state of \spack{} in terms of configuration and installed software. This
directive:
%
\begin{minted}[fontsize=\small, bgcolor=bg]{python}
version('1.2.11', sha256='ah45rstgef...')
\end{minted}
%
in \texttt{zlib}'s recipe is translated to the following fact:
%
\begin{minted}[fontsize=\small, bgcolor=bg]{prolog}
version_declared("zlib", "1.2.11", 0).
\end{minted}
%
where {\tt 0} is the preference weight of this version. Similarly:
%
\begin{minted}[fontsize=\small, bgcolor=bg]{prolog}
zlib@1.2.11
\end{minted}
%
generates the following three facts:
%
\begin{minted}[fontsize=\small, bgcolor=bg]{prolog}
root("zlib").
node("zlib").
version_satisfies("zlib","1.2.11").
\end{minted}
%
stating that \texttt{zlib} is a root node and should satisfy a version requirement. A
typical solve has around $10k-100k$ facts that encode dependencies, variants,
preferences, etc.

The logic program encodes the software model used by \spack{} and only contains
first-order rules, integrity constraints, and optimization objectives. The declarative
nature of ASP makes it easy to enforce certain properties on the solution. For
instance, these three lines ensure that we never have a cyclic dependency in a DAG:
%
\begin{minted}[fontsize=\small, bgcolor=bg]{prolog}
path(A, B) :- depends_on(A, B).
path(A, C) :- path(A, B), depends_on(B, C).
:- path(A, B), path(B, A).
\end{minted}
%
The first rule is the base case: if {\tt A} depends on {\tt B} there is a path from {\tt
  A} to {\tt B}. The second rule defines paths to transitive dependencies recursively:
if there is a path from {\tt A} to {\tt B} and {\tt B} depends on {\tt C}, there is a
path from {\tt A} to {\tt C}. The final line is an \emph{integrity constraint} banning
paths from {\tt A} to {\tt B} and paths from {\tt B} to {\tt A} from occurring together
in a solution.

To give a rough idea of the compactness and expressiveness of the ASP encoding, the
entire logic program for the software model described here is around $800$ lines. The
concretization process is straightforward to follow conceptually within \spack:
\begin{enumerate}
\item Generate facts for all possible dependencies/installs \footnotemark;
\item Send logic program and facts to the solver;
\item Retrieve the best stable model; and
\item Build an {\it optimal} concrete DAG from the model.
\end{enumerate}
\footnotetext{We stress that the logic program changes only when the underlying software
  model changes, as opposed to the generated facts that are different whenever the root
  spec to be concretized or \spack's configuration changes.} The word ``optimal'' is
emphasized since, while rules and integrity constraints fully determine if a solution is
valid, we need optimization targets to select one of the many possible solutions in a
way that fits user's expectations.

A good example to illustrate this point is target selection for DAG nodes. In \spack{}
each node being built has a target microarchitecture associated with it, and we want to
use the best target possible while respecting the constraints coming from the compiler
(for example, {\tt gcc@4.8.3} cannot generate optimized instructions for {\tt skylake}
processors). Previously this required some complicated logic mixed with the rest of the
solve. The introduction of \clingo{} greatly simplified the problem definition. A
cardinality constraint is used to express that the solver must choose one and only one
target per node:
%
\begin{minted}[fontsize=\small, bgcolor=bg]{prolog}
1 { node_target(Package, Target) : target(Target) } 1
    :- node(Package).
\end{minted}
%
A user's choice will force the target of a node:
%
\begin{minted}[fontsize=\small, bgcolor=bg]{prolog}
node_target(P, T) :- node(P), node_target_set(P, T).
\end{minted}
%
An integrity constraint prevents choosing targets not supported by the chosen compiler:
%
\begin{minted}[fontsize=\small, bgcolor=bg]{prolog}
:- node_target(P, T),
   not compiler_supports_target(C, V, T),
   node_compiler(P, C),
   node_compiler_version(P, C, V).
\end{minted}
%
These three statements fully describe the characteristics of a valid solution. To pick
the \emph{optimal} solution we also \emph{weight} the possible targets (the lower the
weight, the best the target) and optimize over the sum of target weights:
%
\begin{minted}[fontsize=\small, bgcolor=bg]{prolog}
node_target_weight(P, W) :-
  node(P), node_target(P, T), target_weight(T, W).
#minimize { W@5,P : node_target_weight(P, W) }.
\end{minted}

\subsection{Generalized Condition Handling}
\label{subsec:generalizedcond}
A unique feature of \spack, as a package manager, is that it optimizes not only for
versions but for many other aspects of the build, e.g., which compiler to use,
which microarchitecture to target, etc. The DSL used for package recipes reflects this
complexity by having multiple directives to declare properties or constraints
on software packages, as seen in Section~\ref{sec:software-model}.

One interesting abstraction that we observed while coding the ASP logic program is
that each of these directives can be seen as a way to impose additional constraints on
the solution, conditional on other constraints being met. For instance, the following
directive in a package:
%
\begin{minted}[fontsize=\small, bgcolor=bg]{python}
depends_on('hdf5+mpi', when='+mpi')
\end{minted}
%
means that, if the spec has the {\tt mpi} variant turned on, then it depends on {\tt hdf5+mpi}. Similarly:
%
\begin{minted}[fontsize=\small, bgcolor=bg]{python}
provides('lapack', when='@12.0:')
\end{minted}
%
means that a package provides the {\tt LAPACK} API if its version is $12.0$ or greater.
This property allowed to encode all the directives as \emph{generalized conditions}, where most of the semantics is encoded abstractly in a few lines of the logic program.

Getting back to a simple example, the snippet below:
%
\begin{minted}[fontsize=\small, bgcolor=bg]{python}
class H5utils(AutotoolsPackage):
    depends_on('png@1.6.0:', when='+png')
\end{minted}
%
is translated to the following facts:
%
\begin{minted}[fontsize=\scriptsize, bgcolor=bg]{prolog}
condition(153).
condition_requirement(153, "node", "h5utils").
condition_requirement(153, "variant_value", "h5utils", "png", "true").
imposed_constraint(153, "version_satisfies", "libpng", "1.6.0:").
dependency_condition(153, "h5utils", "libpng").
\end{minted}
%
when setting up the problem to be solved by \clingo. The important points to note are that:

\begin{itemize}
\item Each directive is associated with a unique global ID.
\item Constraints are either ``requirement'' or ``imposed''.
\item Different type of conditions have different semantics\footnotemark
\end{itemize}
\footnotetext{For instance, the {\tt dependency\_condition} fact is present only for {\tt depends\_on} directives and activates logic that is specific to dependencies.}

The code to trigger and impose general conditions in the logic program is surprisingly simple to read. ASP conditional rules allow us to effectively build new rules from input facts:
%
\begin{minted}[fontsize=\small, bgcolor=bg]{prolog}
condition_holds(ID) :-
 condition(ID);
 attr(N, A1)    : condition_requirement(ID, N, A1);
 attr(N, A1, A2): condition_requirement(ID, N, A1, A2).
\end{minted}
%
based on their number of arguments, or \emph{arity}. Other rules:
%
\begin{minted}[fontsize=\small, bgcolor=bg]{prolog}
attr(N, A1)     :- condition_holds(ID),
                   imposed_constraint(ID, N, A1).
attr(N, A1, A2) :- condition_holds(ID),
                   imposed_constraint(ID, N, A1, A2).
\end{minted}
%
enforce the imposed constraints when a condition holds.

\subsection{Usability Improvements due to \clingo}
As mentioned, the original concretizer was incomplete and could fail to find a solution
when one exists. Users would work around such false negatives by overconstraining
problematic specs, to help the solver find the right answer. This could become very
tedious.

%The drawback was increased
%need for boilerplate code, which became particularly evident in large production
%environments and complex scientific software. With the adoption of \clingo{}
%overspecification is not needed anymore and \emph{false negatives} have been effectively
%eliminated from concretization.

\subsubsection{Conditional Dependencies}
A prominent example of this behavior is with packages having dependencies conditional to a variant being set to a non-default value. Let's take for instance {\tt hpctoolkit}, which has the following directives:

\begin{minted}[fontsize=\small, bgcolor=bg]{python}
class Hpctoolkit(AutotoolsPackage):
    variant('mpi', default=False, description='...')
    depends_on('mpi', when='+mpi')
\end{minted}

Trying to use the old algorithm to concretize {\tt hpctoolkit \^{}mpich} fails like this:

\begin{minted}[fontsize=\footnotesize, bgcolor=bg]{console}
$ spack spec hpctoolkit ^mpich
Input spec
--------------------------------
hpctoolkit
    ^mpich

Concretized
--------------------------------
==> Error: Package hpctoolkit does not depend on mpich
\end{minted}
%
since the greedy algorithm would set variant values before descending to dependencies.
Since no value is specified for the {\tt mpi} variant, the value chosen is
{\tt false} (the default) which leads to {\tt hpctoolkit} having no dependency on
{\tt mpi}. The workaround required users to understand the conditional dependency and
write {\tt hpctoolkit+mpi \^{}mpich} to concretize successfully. With \clingo{}, the
concretizer simply {\it finds} the correct value for the {\tt mpi} variant, as
setting it is the only way for {\tt mpich} to be part of the solution.

%, as shown in Figure~\ref{fig:hpctoolkit}.

% FIXME: Figure is too small, what should we do about that? Remove it? Show only part?
%\begin{figure}[h]
%\includegraphics[width=\columnwidth]{figures/hpctoolkit_concretized.pdf}
%\caption{{\tt hpctoolkit \^{}mpich} as concretized on an Ubuntu 20.04 OS using system compilers}
%\label{fig:hpctoolkit}
%\end{figure}

\subsubsection{Conflicts in Packages}
Before using \clingo{}, conflicts in packages were only used to \emph{validate} a
solution computed by the greedy-algorithm. If the solution matched any conflict,
\spack{} would have errored and hinted the user on how to overconstrain the initial spec
to help concretization. With ASP, conflicts are generalized as constraints during the
solve\footnotemark{} and \spack{} no longer needs to ask the user to be more specific.
\footnotetext{With \clingo{} conflicts are treated as shown in
  Section~\ref{subsec:generalizedcond} and, by imposing \emph{integrity constraints} on
  the problem, they effectively prevent portions of the search space from being
  explored}

\subsubsection{Specialization on Providers of Virtual Packages}
Using \clingo{} also enabled more complex use cases e.g. imposing constraints on
specific virtual package providers. A simple example of that is given by the
{\tt berkeleygw} package, which has the following directives:

\begin{minted}[fontsize=\small, bgcolor=bg]{python}
class Berkeleygw(MakefilePackage):
    variant('openmp', default=True)
    depends_on('lapack')
    depends_on('openblas threads=openmp',
               when='+openmp ^openblas')
\end{minted}

The last directive forces {\tt openblas} to have {\tt openmp} support if
{\tt berkeleygw} has {\tt openmp} support and {\tt openblas} has been chosen as
a provider for the mandatory {\tt lapack} virtual dependency. Conditional constraints
of that complexity could not be expressed before, because the solver would select
defaults for {\tt openblas} before evaluating the conditional constraint on it.

\subsection{Optimization Criteria}

\begin{table}[t]
\centering
\begin{tabular}{| c || l |}
 \hline
 Priority & Criterion (to be minimized) \\
 \hline\hline
 1  & Deprecated versions used \\
 \hline
 2  & Version oldness (roots) \\
 3  & Non-default variant values (roots) \\
 4  & Non-preferred providers (roots) \\
 5  & Unused default variant values (roots) \\
 \hline
 6  & Non-default variant values (non-roots) \\
 7  & Non-preferred providers (non-roots) \\
 \hline
 8  & Compiler mismatches \\
 9  & OS mismatches \\
 10 & Non-preferred OS's \\
 \hline
 11 & Version oldness (non-roots) \\
 12 & Unused default variant values (non-roots) \\
 \hline
 13 & Non-preferred compilers \\
 14 & Target mismatches \\
 15 & Non-preferred targets \\
 \hline
\end{tabular}
\caption{
  Spack's optimization criteria in order of priority.
  \label{table:optimization-criteria}
  \vspace{-1em}
}
\end{table}

The conditional logic presented here provides a great deal of flexibility, but with this
flexibility we also need to have sensible defaults for users who do not want to think
about all of the degrees of freedom Spack allows. Coming up with ``intuitive'' solutions
to the package configuration problem is surprisingly difficult, but we have developed
the list of optimization criteria in Figure~\ref{table:optimization-criteria} based on
our experiences interacting with users and facilities. There are currently 15 criteria
we consult at to pick the best valid solution for a package DAG. All are minimization
criteria, and they are evaluated in lexicographic order, i.e., the highest priority
criteria are optimized first, then the next highest, and so on.

Our top priority (1) is to avoid software versions that have been deprecated due to
security concerns or bugs. Priorities 2-5 choose the newest versions and default variant
values and providers for {\it roots} in the DAG. We prioritize versions first, then
virtual providers (e.g., the user's preferred MPI), then default variant values. After
root configuration, we prioritize non-root configuration (6-7, 11-12). Preferences flow
downward in the DAG via dependency relationships, and if a root demands a particular
version of a dependency, the dependency's preference should not be able to override
this. Ideally, we would enforce strict DAG precedence on preferences (i.e., every node's
preference would dominate those of its dependencies), but we have seen performance
penalties when attempting to track the relative depths of all nodes, so we settle for a
two-level hierarchy of roots and non-roots. We try to enforce consistency in the
resolved graph by minimizing ``mismatches'' (8-9, 14)---these criteria ensure that
neighboring nodes are assigned consistent compilers, OS's and targets unless otherwise
requested. We model preferred compilers, targets, and OS's for non-roots at lower
priority than mismatches, so that dependencies inherit these properties from their
dependents unless set explicitly by the user. All preferred versions, compilers, OS's,
targets, etc. can be overridden through configuration in Spack.

Spack's optimization criteria are unique among dependency solvers in that many of them
are simply not modeled by other systems. In a typical Linux distribution like Debian,
target and toolchain compatibility are not modeled---they are uniform across the DAG. If
we look at {\tt aspcud}~\cite{gebser+:2011-aspcud}, an ASP solver plugin for Debian's
APT, we see that there are around 6 optimization criteria, all to do with
versions and version preferences, far fewer than the 15 criteria modeled here.
%{\tt
%  aspcud} also does not model edges in the graph the same way---it only considers one
%installation of each package, so nodes are either included in the model or not.
