%%%%%%%%%%%%%%%%%%%%%%%%%%%%%%%%%%%%%%%%%%%%%%%%%%%%%%%%%%%%%%%%%%%%%%%%%%%%%%%%
% Template for USENIX papers.
%
% History:
%
% - TEMPLATE for Usenix papers, specifically to meet requirements of
%   USENIX '05. originally a template for producing IEEE-format
%   articles using LaTeX. written by Matthew Ward, CS Department,
%   Worcester Polytechnic Institute. adapted by David Beazley for his
%   excellent SWIG paper in Proceedings, Tcl 96. turned into a
%   smartass generic template by De Clarke, with thanks to both the
%   above pioneers. Use at your own risk. Complaints to /dev/null.
%   Make it two column with no page numbering, default is 10 point.
%
% - Munged by Fred Douglis <douglis@research.att.com> 10/97 to
%   separate the .sty file from the LaTeX source template, so that
%   people can more easily include the .sty file into an existing
%   document. Also changed to more closely follow the style guidelines
%   as represented by the Word sample file.
%
% - Note that since 2010, USENIX does not require endnotes. If you
%   want foot of page notes, don't include the endnotes package in the
%   usepackage command, below.
% - This version uses the latex2e styles, not the very ancient 2.09
%   stuff.
%
% - Updated July 2018: Text block size changed from 6.5" to 7"
%
% - Updated Dec 2018 for ATC'19:
%
%   * Revised text to pass HotCRP's auto-formatting check, with
%     hotcrp.settings.submission_form.body_font_size=10pt, and
%     hotcrp.settings.submission_form.line_height=12pt
%
%   * Switched from \endnote-s to \footnote-s to match Usenix's policy.
%
%   * \section* => \begin{abstract} ... \end{abstract}
%
%   * Make template self-contained in terms of bibtex entires, to allow
%     this file to be compiled. (And changing refs style to 'plain'.)
%
%   * Make template self-contained in terms of figures, to
%     allow this file to be compiled.
%
%   * Added packages for hyperref, embedding fonts, and improving
%     appearance.
%
%   * Removed outdated text.
%
%%%%%%%%%%%%%%%%%%%%%%%%%%%%%%%%%%%%%%%%%%%%%%%%%%%%%%%%%%%%%%%%%%%%%%%%%%%%%%%%

\documentclass[letterpaper,twocolumn,10pt]{article}
\usepackage{usenix-2020-09}

% to be able to draw some self-contained figs
\usepackage{tikz}
\usepackage{amsmath}
\usepackage{subfig}

% use for decent code formatting
\usepackage{inconsolata}
\usepackage[outputdir=build]{minted}


% notes for different people
\newcommand{\tg}[1]{\ifisdraft{\color{blue}[#1 -- Todd]}\fi}
\newcommand{\spack}{\texttt{Spack}}
\newcommand{\clingo}{\texttt{clingo}}
\definecolor{bg}{rgb}{0.95,0.95,0.95}
%-------------------------------------------------------------------------------
\begin{document}
%-------------------------------------------------------------------------------

%don't want date printed
\date{}

% make title bold and 14 pt font (Latex default is non-bold, 16 pt)
\title{\Large \bf Formatting Submissions for a USENIX Conference:\\
  An (Incomplete) Example}

%for single author (just remove % characters)
\author{
{\rm Your N.\ Here}\\
Your Institution
\and
{\rm Second Name}\\
Second Institution
% copy the following lines to add more authors
% \and
% {\rm Name}\\
%Name Institution
} % end author

\maketitle

\begin{abstract}
  HPC and AI software has become extremely complex. Software packages make use of many
  programming models and libraries to exploit a growing variety of available GPUs and
  other accelerators. Package managers can mitigate this complexity, but integrating a
  coherent software stack has never been harder. Simply finding compatible dependency
  {\it versions} NP-complete, and with build-time options, multiple GPUs, flags, and
  other parameters, the configuration space is enormous. Simply {\it modeling}
  compatibility semantics to choose a ``good'' configuration is daunting.

  We tackle this complexity using Answer Set Programming (ASP), a declarative model for
  combinatorial search problems. We integrate an ASP solver into the popular {\it Spack}
  package manager, and we show how it allows us to concisely model ever-growing
  complexities of HPC software, while seamlessly mixing source builds with optimized
  binaries. We show good performance on repositories with tens of thousands of package
  builds.
\end{abstract}

%-------------------------------------------------------------------------------
\section{Introduction}
%-------------------------------------------------------------------------------
\label{sec:intro}

Managing dependencies for scientific software has been notoriously difficult for
years~\cite{hoste+:pyhpc12,gamblin+:sc15,dubois2003johnny,hochstein+:2011-build}.
Scientific software is typically built from source to achieve the best performance, and
configuring build systems to target speicfic machines, dependency versions, and
compilers requires painstaking care if done manually. In the early 2010's, two main
package managers emerged to tackle this problem for the HPC space:
Spack~\cite{gamblin+:sc15} and EasyBuild~\cite{hoste+:pyhpc12}.


\cite{}

E4S is ~100 packages and 400 dependencies
many conditional dependencies
many version constraints
many optionally enabled packages
many virtual packages






Our contributions are:

\begin{enumerate}
\item A general mapping of HPC dependency compatibility semantics and optimization rules to ASP;
\item A technique to optimize for reuse of existing builds in combinatorial package solves;
\item An implementation of ASP solver in {\it Spack}, a popular HPC package manager; and
\item An evaluation of our system's performance on large, real-world HPC package repositories.
\end{enumerate}

%-------------------------------------------------------------------------------
\section{Package Managers}
%-------------------------------------------------------------------------------
\label{sec:pack-managers}
-- Todd
Background on different kinds of package managers, highlighting commonalities and differences (e.g. conda, Go, system package managers). Introduce \spack and explain how it is different from others.


%-------------------------------------------------------------------------------
\section{\spack}
\label{sec:software-model}
%-------------------------------------------------------------------------------
-- Greg
\subsection{Spec Syntax}

\subsection{Package DSL}

\subsection{Concretization}
- user configuration
- CLI arguments
- package constraints

\subsection{Dependency Model}
Discuss the software dependency model used in \spack and the future roadmap. Introduce concepts like variants, ABI compatibility etc. that are either implicit or disregarded by other package managers.


%-------------------------------------------------------------------------------
\section{Answer Set Programming}
%-------------------------------------------------------------------------------
-- Todd
Background on ASP solvers

%-------------------------------------------------------------------------------
\section{Modeling Software Dependencies with ASP}
%-------------------------------------------------------------------------------
\label{sec:asp-model}
%-------------------------------------------------------------------------------
\section{Modeling Software Dependencies with ASP}
%-------------------------------------------------------------------------------
\label{sec:asp-model}

At a high-level, using \clingo, our concretizer combines:
\begin{enumerate}
\item Many \emph{facts} characterizing the problem instance;
\item A small \emph{logic program} encoding the rules and constraints of the software model; and
\item Optimization criteria that define an {\it optimal} model.
\end{enumerate}
The facts are always generated starting from one or more \emph{root specs}, and they
account for metadata from all package recipes of possible dependencies, as well as
the current state of \spack{} in terms of configuration and installed software. This
directive:
%
\begin{minted}[fontsize=\small, bgcolor=bg]{python}
version('1.2.11', sha256='ah45rstgef...')
\end{minted}
%
in \texttt{zlib}'s recipe is translated to the following fact:
%
\begin{minted}[fontsize=\small, bgcolor=bg]{prolog}
version_declared("zlib", "1.2.11", 0).
\end{minted}
%
where {\tt 0} is the preference weight of this version. Similarly, the request to
concretize
%
\begin{minted}[fontsize=\small, bgcolor=bg]{prolog}
zlib@1.2.11
\end{minted}
%
generates the following three facts:
%
\begin{minted}[fontsize=\small, bgcolor=bg]{prolog}
root("zlib").
node("zlib").
version_satisfies("zlib","1.2.11").
\end{minted}
%
stating that \texttt{zlib} is a root node and should satisfy a version requirement. A
typical solve has around $10k-100k$ facts that encode dependencies, variants,
preferences, etc.

The logic program encodes the software model used by \spack{} and only contains
first-order rules, integrity constraints, and optimization objectives. The declarative
nature of ASP makes it easy to enforce certain properties on the solution. For
instance, these three lines ensure that we never have a cyclic dependency in a DAG:
%
\begin{minted}[fontsize=\small, bgcolor=bg]{prolog}
path(A, B) :- depends_on(A, B).
path(A, C) :- path(A, B), depends_on(B, C).
:- path(A, B), path(B, A).
\end{minted}
%
The first rule is the base case: if {\tt A} depends on {\tt B} there is a path from {\tt
  A} to {\tt B}. The second rule defines paths to transitive dependencies recursively:
if there is a path from {\tt A} to {\tt B} and {\tt B} depends on {\tt C}, there is a
path from {\tt A} to {\tt C}. The final line is an \emph{integrity constraint} banning
paths from {\tt A} to {\tt B} and paths from {\tt B} to {\tt A} from occurring together
in a solution.

To give a rough idea of the compactness and expressiveness of the ASP encoding, the
entire logic program for the software model described here is around $700$ lines. The
concretization process is straightforward to follow conceptually within \spack:
\begin{enumerate}
\item Generate facts for all possible dependencies/installs \footnotemark;
\item Send logic program and facts to the solver;
\item Retrieve the best stable model; and
\item Build an {\it optimal} concrete DAG from the model.
\end{enumerate}
\footnotetext{We stress that the logic program changes only when the underlying software
  model changes, as opposed to the generated facts that are different whenever the root
  spec to be concretized or \spack's configuration changes.} The word ``optimal'' is
emphasized since, while rules and integrity constraints fully determine if a solution is
valid, we need optimization targets to select one of the many possible solutions in a
way that fits user's expectations.

A good example to illustrate this point is target selection for DAG nodes. In \spack{}
each node being built has a target microarchitecture associated with it, and we want to
use the best target possible while respecting the constraints coming from the compiler
(for example, {\tt gcc@4.8.3} cannot generate optimized instructions for {\tt skylake}
processors). Previously this required some complicated logic mixed with the rest of the
solve. The introduction of \clingo{} greatly simplified the problem definition. A
cardinality constraint is used to express that each node must have one and only one
target:
%
\begin{minted}[fontsize=\small, bgcolor=bg]{prolog}
1 { node_target(Package, Target) :
    target(Target) } 1 :- node(Package).
\end{minted}
%
A rule ensures that the user's choice to set the target to a specific value is respected:
%
\begin{minted}[fontsize=\small, bgcolor=bg]{prolog}
node_target(P, T) :- node(P), node_target_set(P, T).
\end{minted}
%
An integrity constraints prevents choosing a target not supported by the current compiler:
%
\begin{minted}[fontsize=\small, bgcolor=bg]{prolog}
:- node_target(P, T),
   not compiler_supports_target(C, V, T),
   node_compiler(P, C),
   node_compiler_version(P, C, V).
\end{minted}
%
These three statements fully describe the characteristics of a valid solution. To pick
the \emph{optimal} solution we also \emph{weight} the possible targets (the lower the
weight, the best the target) and optimize over the sum of target weights:
%
\begin{minted}[fontsize=\small, bgcolor=bg]{prolog}
node_target_weight(P, W) :-
  node(P), node_target(P, T), target_weight(T, W).
#minimize { W@5,P : node_target_weight(P, W) }.
\end{minted}

\subsection{Generalized Condition Handling}
\label{subsec:generalizedcond}
A unique feature of \spack, as a package manager, is that it doesn't only optimize for
versions but for many other aspects of the build as well e.g. which compiler to use,
which microarchitecture to target, etc. The DSL used for package recipes reflects this
complexity by having multiple directives to declare different properties or constraints
for each software, as seen in Section~\ref{sec:software-model}.

One interesting abstraction, that we observed while coding the ASP logic program, is
that each of these directives can be seen as a way to impose additional constraints on
the solution, conditional on other constraints being met. For instance, the following
directive in a package:
%
\begin{minted}[fontsize=\small, bgcolor=bg]{python}
depends_on('hdf5+mpi', when='+mpi')
\end{minted}
%
means that, if the spec has the {\tt mpi} variant turned on, then it depends on {\tt hdf5+mpi}. Similarly:
%
\begin{minted}[fontsize=\small, bgcolor=bg]{python}
provides('lapack', when='@12.0:')
\end{minted}
%
means that a package provides the {\tt LAPACK} API if its version is $12.0$ or greater.
This property allowed to encode all the directives as \emph{generalized conditions}, where most of the semantics is encoded abstractly in a few lines of the logic program.

Getting back to a simple example, the snippet below:
%
\begin{minted}[fontsize=\small, bgcolor=bg]{python}
class H5utils(AutotoolsPackage):
    depends_on('png@1.6.0:', when='+png')
\end{minted}
%
is translated to the following facts:
%
\begin{minted}[fontsize=\scriptsize, bgcolor=bg]{prolog}
condition(153).
condition_requirement(153, "node", "h5utils").
condition_requirement(153, "variant_value", "h5utils", "png", "true").
imposed_constraint(153, "version_satisfies", "libpng", "1.6.0:").
dependency_condition(153, "h5utils", "libpng").
\end{minted}
%
when setting up the problem to be solved by \clingo. The important points to note are that:

\begin{itemize}
\item Each directive is associated with a unique global ID.
\item Constraints are either ``requirement'' or ``imposed''.
\item Different type of conditions have different semantics\footnotemark
\end{itemize}
\footnotetext{For instance, the {\tt dependency\_condition} fact is present only for {\tt depends\_on} directives and activates logic that is specific to dependencies.}

The code to trigger and impose general conditions in the logic program is surprisingly simple to read. ASP conditional rules allow us to effectively build new rules from input facts:
%
\begin{minted}[fontsize=\small, bgcolor=bg]{prolog}
condition_holds(ID) :-
condition(ID);
attr(N, A1): condition_requirement(ID, N, A1);
attr(N, A1, A2): condition_requirement(ID, N, A1, A2);
...
\end{minted}
%
based on their \emph{arity}. Other rules, similar to:
%
\begin{minted}[fontsize=\small, bgcolor=bg]{prolog}
attr(N, A1) :-
  condition_holds(ID), imposed_constraint(ID, N, A1).
\end{minted}
%
enforce the imposed constraints when a condition holds.

\subsection{Usability Improvements due to \clingo}
As mentioned, the original concretizer was incomplete and could fail to find a solution
when one exists. Users would work around such false negatives by overconstraining
problematic specs, to help the solver find the right answer. This could become very
tedious.

%The drawback was increased
%need for boilerplate code, which became particularly evident in large production
%environments and complex scientific software. With the adoption of \clingo{}
%overspecification is not needed anymore and \emph{false negatives} have been effectively
%eliminated from concretization.

\subsubsection{Conditional Dependencies}
A prominent example of this behavior is with packages having dependencies conditional to a variant being set to a non-default value. Let's take for instance {\tt hpctoolkit}, which has the following directives:

\begin{minted}[fontsize=\small, bgcolor=bg]{python}
class Hpctoolkit(AutotoolsPackage):
    variant('mpi', default=False, description='...')
    depends_on('mpi', when='+mpi')
\end{minted}

Trying to use the old algorithm to concretize {\tt hpctoolkit \^{}mpich} fails like this:

\begin{minted}[fontsize=\footnotesize, bgcolor=bg]{console}
% spack spec hpctoolkit ^mpich
Input spec
--------------------------------
hpctoolkit
    ^mpich

Concretized
--------------------------------
==> Error: Package hpctoolkit does not depend on mpich
\end{minted}
%
since the greedy algorithm would set variant values before descending to dependencies.
Since no value is specified for the {\tt mpi} variant, the value chosen is
{\tt false} (the default) which leads to {\tt hpctoolkit} having no dependency on
{\tt mpi}. The workaround required users to understand the conditional dependency and
write {\tt hpctoolkit+mpi \^{}mpich} to concretize succesfully. With \clingo{}, the
concretizer simply {\it finds} the correct value for the {\tt mpi} variant, as
setting it is the only way for {\tt mpich} to be part of the solution.

%, as shown in Figure~\ref{fig:hpctoolkit}.

% FIXME: Figure is too small, what should we do about that? Remove it? Show only part?
%\begin{figure}[h]
%\includegraphics[width=\columnwidth]{figures/hpctoolkit_concretized.pdf}
%\caption{{\tt hpctoolkit \^{}mpich} as concretized on an Ubuntu 20.04 OS using system compilers}
%\label{fig:hpctoolkit}
%\end{figure}

\subsubsection{Conflicts in Packages}
Before using \clingo{}, conflicts in packages were only used to \emph{validate} a
solution computed by the greedy-algorithm. If the solution matched any conflict,
\spack{} would have errored and hinted the user on how to overconstrain the initial spec
to help concretization. With ASP, conflicts are generalized as constraints during the
solve\footnotemark{} and \spack{} no longer needs to ask the user to be more specific.
\footnotetext{With \clingo{} conflicts are treated as shown in
  Section~\ref{subsec:generalizedcond} and, by imposing \emph{integrity constraints} on
  the problem, they effectively prevent portions of the search space from being
  explored}

\subsubsection{Specialization on Providers of Virtual Packages}
Using \clingo{} also enabled more complex use cases e.g. imposing constraints on
specific virtual package providers. A simple example of that is given by the
{\tt berkeleygw} package, which has the following directives:

\begin{minted}[fontsize=\small, bgcolor=bg]{python}
class Berkeleygw(MakefilePackage):
    variant('openmp', default=True)
    depends_on('lapack')
    depends_on('openblas threads=openmp',
               when='+openmp ^openblas')
\end{minted}

The last directive forces {\tt openblas} to have {\tt openmp} support if
{\tt berkeleygw} has {\tt openmp} support and {\tt openblas} has been chosen as
a provider for the mandatory {\tt lapack} virtual dependency. Conditional constraints
of that complexity could not be expressed before, because the solver would select
defaults for {\tt openblas} before evaluating the conditional constraint on it.

\subsection{Optimization Criteria}

\begin{table}[t]
\centering
\begin{tabular}{| c || l |}
 \hline
 Priority & Criterion (to be minimized) \\
 \hline\hline
 1  & Deprecated versions used \\
 \hline
 2  & Version oldness (roots) \\
 3  & Non-default variant values (roots) \\
 4  & Non-preferred providers (roots) \\
 5  & Unused default variant values (roots) \\
 \hline
 6  & Non-default variant values (non-roots) \\
 7  & Non-preferred providers (non-roots) \\
 \hline
 8  & Compiler mismatches \\
 9  & OS mismatches \\
 10 & Non-preferred OS's \\
 \hline
 11 & Version oldness (non-roots) \\
 12 & Unused default variant values (non-roots) \\
 \hline
 13 & Non-preferred compilers \\
 14 & Target mismatches \\
 15 & Non-preferred targets \\
 \hline
\end{tabular}
\caption{
  Spack's optimization criteria in order of priority.
  \label{table:optimization-criteria}
  \vspace{-1em}
}
\end{table}

The conditional logic presented here provides a great deal of flexibility, but with this
flexibility we also need to have sensible defaults for users who do not want to think
about all of the degrees of freedom Spack allows. Coming up with ``intuitive'' solutions
to the package configuration problem is surprisingly difficult, but we have developed
the list of optimization criteria in Figure~\ref{table:optimization-criteria} based on
our experiences interacting with users and facilities. There are currently 15 criteria
we consult at to pick the best valid solution for a package DAG. All are minimization
criteria, and they are evaluated in lexicogrpahic order, i.e., the highest priority
criteria are optimized first, then the next highest, and so on.

Our top priority (1) is to avoid software versions that have been deprecated due to
security concerns or bugs. Priorities 2-5 choose the newest versions and default variant
values and providers for {\it roots} in the DAG. We prioritize versions first, then
virtual providers (e.g., the user's preferred MPI), then default variant values. After
root configuration, we prioritize non-root configuration (6-7, 11-12). Preferences flow
downward in the DAG via dependency relationships, and if a root demands a particular
version of a dependency, the dependency's preference should not be able to override
this. Ideally, we would enforce strict DAG precedence on preferences (i.e., every node's
preference would dominate those of its dependencies), but we have seen performance
penalties when attempting to track the relative depths of all nodes, so we settle for a
two-level hierarchy of roots and non-roots. We try to enforce consistency in the
resolved graph by minimizing ``mismatches'' (8-9, 14)---these criteria ensure that
neighboring nodes are assigned consistent compilers, OS's and targets unless otherwise
requested. We model preferred compilers, targets, and OS's for non-roots at lower
priority than mismatches, so that dependencies inherit these properties from their
dependents unless set explicitly by the user. All preferred versions, compilers, OS's,
targets, etc. can be overridden through configuration in Spack.

Spack's optimization criteria are unique among dependency solvers in that many of them
are simply not modeled by other systems. In a typical Linux distribution like Debian,
target and toolchain compatibility are not modeled---they are uniform across the DAG. If
we look at {\tt aspcud}~\cite{gebser+:2011-aspcud}, an ASP solver plugin for Debian's
APT, we see that there are around 6 optimization criteria, but they all have to do with
versions and version preferences, far fewer than the 15 criteria modeled here. {\tt
  aspcud} also does not model edges in the graph the same way---it only considers one
installation of each package, so nodes are either included in the model or not.




\section{Reusing installed packages}
Max / Todd
\subsection{reuse in nix, guix, conan, and old \spack}
- by hash

\subsection{optimizing for reuse}
- reuse of half the generalized condition
- Reusing built dependencies

\subsection{optimizing for reuse}
- architecting optimization criteria for builds vs. reuse

%-------------------------------------------------------------------------------
\section{Error Handling for Unsolvable Problems}
%-------------------------------------------------------------------------------
Greg
How do we want to perform error handling?


%-------------------------------------------------------------------------------
\section{Performance Results}
%-------------------------------------------------------------------------------
\label{sec:perf-results}
%-------------------------------------------------------------------------------
\section{Performance Results}
%-------------------------------------------------------------------------------
\label{sec:perf-results}

% How is the ASP based solver performing?

The \clingo{} solver performance given a logic program depends on a number of factors.
First, the number of facts in a specific concretization. Second, the configuration and
various optimization parameters passed to the solver.

The solving process consists of four stages: \emph{setup}, \emph{load}, \emph{ground},
and \emph{solve}. The first two are preliminary phases and the other two perform the
solve. Specifically, the setup phase generates the facts for the given spec,
whereas the load phase loads the main logic program (i.e., the rules of the
software model) as a resource into the solver. The grounding phase comes first. Once we
have a grounded program, we can run the last phase: the full solve in \clingo{}.
%
We instrumented the solver to measure each phase.

% 
\begin{figure*}[htb]

    \centering
    \subfloat[][Ground times]{
    \label{subfig:deps_quartz_load}
        \includegraphics[width=\perfsubfigwidth\textwidth]{figures/perf/deps_quartz_ground_fig.pdf}
    }\hfill%
    \subfloat[][Solve times]{
    \label{subfig:deps_quartz_solve}
        \includegraphics[width=\perfsubfigwidth\textwidth]{figures/perf/deps_quartz_solve_fig.pdf}
    }\hfill%
    \subfloat[][Full solving times]{
    \label{subfig:deps_quartz_full}
        \includegraphics[width=\perfsubfigwidth\textwidth]{figures/perf/deps_quartz_total_fig.pdf}
    }
    \caption{Solve times vs. number of dependent packages across all packages on the Quartz machine.}
    \label{fig:deps_quartz}

\end{figure*}


% 
\begin{figure*}[htb]

    \centering
    \subfloat[][Load times]{
    \label{subfig:deps_lassen_load}
        \includegraphics[width=0.32\textwidth]{figures/perf/deps_lassen_load_fig.pdf}
    }\hfill%
    \subfloat[][Solve times]{
    \label{subfig:deps_lassen_solve}
        \includegraphics[width=0.32\textwidth]{figures/perf/deps_lassen_solve_fig.pdf}
    }\hfill%
    \subfloat[][Full solving times]{
    \label{subfig:deps_lassen_full}
        \includegraphics[width=0.32\textwidth]{figures/perf/deps_lassen_total_fig.pdf}
    }
    \caption{Solve times vs. number of dependent packages across all packages on the Lassen machine.}
    \label{fig:deps_quartz}

\end{figure*}


\begin{figure*}[htb]

    \centering
    \subfloat[][Ground times vs. number of dependencies.]{
    \label{subfig:deps_quartz_load_a}
        \includegraphics[width=\perfsubfigwidth\textwidth]{figures/new-quartz/deps_quartz_ground_fig.pdf}
    }\hfill%
    \subfloat[][Solve times vs. number of dependencies.]{
    \label{subfig:deps_quartz_solve_a}
        \includegraphics[width=\perfsubfigwidth\textwidth]{figures/new-quartz/deps_quartz_solve_fig.pdf}
    }\hfill%
    \subfloat[][Total times vs. number of dependencies.]{
    \label{subfig:deps_quartz_full_a}
        \includegraphics[width=\perfsubfigwidth\textwidth]{figures/new-quartz/deps_quartz_total_fig.pdf}
    }\hfill%
    \subfloat[][CDF of solve time for different {\tt clingo} configurations.]{
    \label{subfig:cdf_quartz_full_a}
        \includegraphics[width=\perfsubfigwidth\textwidth]{figures/new-quartz/cdf_quartz_total_fig.pdf}
    }\\
    \subfloat[][CDF of setup times for different cache sizes across all E4S packages.]{
    \label{subfig:cdf_e4s_quartz_load_a}
        \includegraphics[width=\perfsubfigwidth\textwidth]{figures/new-quartz/cdf_e4s_cache_quartz_setup_fig.pdf}
    }\hfill%
    \subfloat[][CDF of solve times for different cache sizes across all E4S packages.]{
    \label{subfig:cdf_e4s_quartz_solve_a}
        \includegraphics[width=\perfsubfigwidth\textwidth]{figures/new-quartz/cdf_e4s_cache_quartz_solve_fig.pdf}
    }\hfill%
    \subfloat[][CDF of total time for different cache sizes across all E4S packages.]{
    \label{subfig:cdf_e4s_quartz_full_a}
        \includegraphics[width=\perfsubfigwidth\textwidth]{figures/new-quartz/cdf_e4s_cache_quartz_total_fig.pdf}
    }\hfill%
    \subfloat[][CDF of old concretizer times and \clingo{} total solve times.]{
    \label{subfig:cdf_old_vs_new_a}

    \includegraphics[width=\perfsubfigwidth\textwidth]{figures/new-quartz/cdf_quartz_old_vs_clingo.pdf}
    }

    \caption{Performance figures of solving times across all packages on Quartz.\vspace{-1em}}
    \label{fig:all_perf}

\end{figure*}


\subsection{System Setup}

All performance runs were executed on a single node each of the Quartz and Lassen
supercomputers at at Lawrence Livermore National Laboratory (LLNL)~\cite{llnl:hpc}.
Quartz is an Intel-based 3.3 PF machine, where each node comprises two Intel Xeon
E5-2695 v4 (Haswell) processors and 128GB of memory. The Lassen machine is a smaller
variant of Sierra, a 125 PF capability supercomputer at LLNL. Each node on Lassen has
two IBM Power9 little-endian processors and four NVIDIA Tesla V100 (Volta) GPUs. There
are 256GB of memory on each node. No hardware accelerators were used in any of our
testing. Experiments ran with NFSv3.

\subsection{Solve timings for all packages}

In this section we examine the solving times for all the packages. First, we focus on
the relation between the solving times and the number of dependencies for each package.

For number of dependencies, we measured the number of possible
dependencies added to the solve, rather than the total dependencies in
the result, because it is a closer measure of the necessary work for
the solver. This leads to natural clustering in the number of
dependencies, as many simple dependencies lead to large numbers of
potential dependencies that dwarf other differences between specs.

Figures~\ref{subfig:deps_quartz_load_a}, \ref{subfig:deps_quartz_solve_a}, and
~\ref{subfig:deps_quartz_full_a} show the grounding, solve, and full solving (i.e.,
involving all the stages) times for all the packages on Quartz. The results on Lassen
are comparable to Quartz and are omitted to save space. Load times, as one would expect,
were not affected by the number of packages. Also, the setup times are the same order of
magnitude as ground times and do not depend on \clingo{}'s performance so they were
omitted in favor of showing times that are directly dependent on \clingo{}. We used the
\clingo{}'s ``tweety'' configuration and ``usc,one'' optimization strategy for running
the solving process. Further below we explore the differences in solving times between
these different strategies.

We can see from the figures that the time increases as the number of possible package
dependencies increases. This is because increased number of possible dependencies leads
to a larger number of facts and a bigger logic program overall. We measure possible
dependencies rather than actual dependencies of the result because possible dependencies
better measure the size of the problem space. In particular, when many packages can
provide a virtual dependency like the Message Passing Interface (MPI), much of the
potential solve space is not present in the final result when a single MPI
implementation is chosen.

Figure~\ref{subfig:deps_quartz_full_a} also shows that there are two major clusters in
the execution times. The clusters are separated by a gap in the possible dependencies.
One cluster contains packages with less than 200 possible dependencies, whereas, the
other major cluster contains packages with more than 400 possible dependencies. This
natural clustering occurs because some low-level dependencies have options that can
trigger huge potential dependency trees, and the gap is between packages that can
include those dependencies and those that cannot. For example, any package that can
depend on MPI in any possible configuration of it and its dependencies involves at least
452 possible dependencies. Since many unusual but technically valid package
configurations can create circular dependencies between MPI and build tools (e.g.
\texttt{mpilander} provides MPI and \texttt{mpilander -> cmake -> qt -> valgrind ->
  mpi}), in practice a gap forms between the packages that (mostly) all can depend on
MPI, and those that cannot. While the solver excludes the configurations that actually
produce circular dependencies, these cycles expand the solution space of the solver and
therefore affect performance.

% TODO: Can we confirm this?? My explanation above is slightly vague -- GBB

% The gap occurs because of dependency on \emph{cmake}, which itself depends
% cumulatively on more than 400 packages.

% 
\begin{figure*}[htb]

    \centering
    \subfloat[][Ground times]{
    \label{subfig:cdf_quartz_ground}
        \includegraphics[width=\perfsubfigwidth\textwidth]{figures/perf/cdf_quartz_ground_fig.pdf}
    }\hfill%
    \subfloat[][Solve times]{
    \label{subfig:cdf_quartz_solve}
        \includegraphics[width=\perfsubfigwidth\textwidth]{figures/perf/cdf_quartz_solve_fig.pdf}
    }\hfill%
    \subfloat[][Full solving times]{
    \label{subfig:cdf_quartz_full}
        \includegraphics[width=\perfsubfigwidth\textwidth]{figures/perf/cdf_quartz_total_fig.pdf}
    }
    \caption{Cumulative distribution of solve times across all packages on the Quartz machine.}
    \label{fig:cdf_quartz}

\end{figure*}


% 
\begin{figure}[htb]

    \centering
    \includegraphics[width=0.3\textwidth]{figures/perf/cdf_quartz_old_vs_clingo.pdf}
    \caption{Cumulative distribution of old concretizer concretization times and \clingo{} solve times across all packages on Quartz.}
    \label{fig:cdf_quartz_old_vs_new}

\end{figure}


% 
\begin{figure*}[htb]

    \centering
    \subfloat[][Ground times]{
    \label{subfig:cdf_lassen_ground}
        \includegraphics[width=0.32\textwidth]{figures/perf/cdf_lassen_ground_fig.pdf}
    }\hfill%
    \subfloat[][Solve times]{
    \label{subfig:cdf_lassen_solve}
        \includegraphics[width=0.32\textwidth]{figures/perf/cdf_lassen_solve_fig.pdf}
    }\hfill%
    \subfloat[][Full solving times]{
    \label{subfig:cdf_lassen_full}
        \includegraphics[width=0.32\textwidth]{figures/perf/cdf_lassen_total_fig.pdf}
    }
    \caption{Cumulative distribution of solve times across all packages on the Lassen machine.}
    \label{fig:cdf_lassen}

\end{figure*}


Besides dependencies another set of factors that influences the execution times are
\clingo{} parameters. Specifically, \clingo{} defines six configuration presets:
\emph{frumpy}, \emph{jumpy}, \emph{tweety}, \emph{trendy}, \emph{crafty}, and
\emph{handy}. Each preset sets numerous low level parameters that control different
aspects of the solver. In our performance study, we specifically focus on three
configurations: \emph{tweety} -- geared towards typical ASP programs, \emph{trendy} --
geared towards industrial problems, and \emph{handy} -- geared towards large problems.

%Figures~\ref{subfig:cdf_quartz_ground_a}, \ref{subfig:cdf_quartz_solve_a}, and~\ref{subfig:cdf_quartz_full_a} show the cumulative distribution of the solve times under \emph{tweety}, \emph{trendy}, and \emph{handy} configurations on Quartz.
Figure~\ref{subfig:cdf_quartz_full_a} shows the cumulative distribution of full solve
times \emph{tweety}, \emph{trendy}, and \emph{handy} configurations on Quartz. The
results on Lassen are comparable to Quartz and are omitted to save space. The vast
majority of packages are fully solved in under 25 seconds on both machines. We also saw
(figure omitted) that there is no difference in ground times between the different
configurations. This suggests that most low level parameters that are tweaked by each
configuration control the actual solving phase. The figures clearly indicate that
\emph{tweety} performs better than the other configurations we benchmarked. This is,
therefore, the default configuration used in the concretization process.

Figure~\ref{subfig:cdf_old_vs_new_a} shows the the cumulative distribution of the old
concretizer concretization times and \clingo{} total solve times (under \emph{tweety}
configuration) on Quartz. Figure~\ref{subfig:deps_quartz_full_a} shows us that about
2.2K packages belong to the cluster with less than 200 dependencies, which means that
the dependency trees of these packages are smaller. This leads to shorter ground and
solve times and that makes \clingo{}'s times correspond to old concretizer times for
these packages. The packages in the other cluster have potentially huge dependency trees
that increase the total solving times and that is reflected in the deviation of
\clingo{}'s times in Figure~\ref{subfig:cdf_old_vs_new_a} from the old concretizer
times.

% - other properties?
% - --single-shot vs. no single-shot?
% - different tactics?

\subsection{Solve timings for all packages with reuse}

In this subsection, we examine the performance of the solver with the \emph{reuse} flag
switched on. As described in Section~\ref{sec:reuse}, reusing packages in a buildcache
increases the number of facts proportionally to the number of cached packages.

% 
\begin{figure*}[htb]

    \centering
    \subfloat[][Setup times]{
    \label{subfig:cdf_e4s_quartz_load}
        \includegraphics[width=0.32\textwidth]{figures/perf/cdf_e4s_cache_quartz_setup_fig.pdf}
    }\hfill%
    \subfloat[][Solve times]{
    \label{subfig:cdf_e4s_quartz_solve}
        \includegraphics[width=0.32\textwidth]{figures/perf/cdf_e4s_cache_quartz_solve_fig.pdf}
    }\hfill%
    \subfloat[][Total solver times]{
    \label{subfig:cdf_e4s_quartz_full}
        \includegraphics[width=0.32\textwidth]{figures/perf/cdf_e4s_cache_quartz_total_fig.pdf}
    }
    \caption{Cumulative distribution of solve times for different cache sizes across all E4S packages on the Quartz machine.}
    \label{fig:cdf_e4s_quartz}

\end{figure*}


% 
\begin{figure*}[htb]

    \centering
    \subfloat[][Setup times]{
    \label{subfig:cdf_e4s_lassen_load}
        \includegraphics[width=0.32\textwidth]{figures/perf/cdf_e4s_cache_lassen_setup_fig.pdf}
    }\hfill%
    \subfloat[][Solve times]{
    \label{subfig:cdf_e4s_lassen_solve}
        \includegraphics[width=0.32\textwidth]{figures/perf/cdf_e4s_cache_lassen_solve_fig.pdf}
    }\hfill%
    \subfloat[][Total solver times]{
    \label{subfig:cdf_e4s_lassen_full}
        \includegraphics[width=0.32\textwidth]{figures/perf/cdf_e4s_cache_lassen_total_fig.pdf}
    }
    \caption{Cumulative distribution of solve times for different cache sizes across all E4S packages on the Lassen machine.}
    \label{fig:cdf_e4s_quartz}

\end{figure*}


We specifically focus on the packages in the ECP Extreme-scale Scientific Software Stack
(E4S) project~\cite{e4s}. It is a community effort to provide open source software
packages for developing, deploying, and running scientific applications on HPC
platforms. There are just under 600 packages in E4S, but the buildcache of the project
targets different architectures, operating systems, and compilers, thereby totaling over
60K pre-compiled packages (hash signatures). We divided the buildcache into 4 groups:
full buildcache (63099 packages), buildcache restricted to the \texttt{ppc64le}
architecture (27160 packages), buildcache restricted to the \texttt{rhel7} OS (15255
packages), and buildcache restricted to both \texttt{ppc64le} architecture and the
\texttt{rhel7} OS (6804 packages). Benchmarking across an increasing size of the
buildcache provides us with a better understanding of the impact of reuse optimization.


Figures~\ref{subfig:cdf_e4s_quartz_load_a}, \ref{subfig:cdf_e4s_quartz_solve_a},
and~\ref{subfig:cdf_e4s_quartz_full_a} show the cumulative distribution of the solve
times of the E4S packages with increasing buildcache on Quartz. The results on Lassen
are comparable to Quartz and are omitted to save space. Setup times are higher than
solve times, even for smaller buildcaches. This happens because when we reuse packages
we need to load the database of existing packages. This is currently time consuming
because it requires us to read and compare many spec objects in Python. There a jump in
the solve times for the largest buildcache, but most solves take less than 10 seconds.
The fact that the runtime is dominated by serial setup time is good news; setup time is
easily optimized away through caching and optimizing Python code, while solve
time is not. The E4S buildcache, with nearly 64,000 packages, is much larger than most
package repositories, and we expect that our approach will scale to much larger
buildcaches if we can optimize the Python runtime. Multi-shot solver techniques may
offer additional solver performance, as we can divide and conquer for a slightly less
optimal final result.

% \begin{figure*}[htb]

    \centering
    \subfloat[][Ground times vs. number of dependencies.]{
    \label{subfig:deps_quartz_load_a}
        \includegraphics[width=\perfsubfigwidth\textwidth]{figures/new-quartz/deps_quartz_ground_fig.pdf}
    }\hfill%
    \subfloat[][Solve times vs. number of dependencies.]{
    \label{subfig:deps_quartz_solve_a}
        \includegraphics[width=\perfsubfigwidth\textwidth]{figures/new-quartz/deps_quartz_solve_fig.pdf}
    }\hfill%
    \subfloat[][Total times vs. number of dependencies.]{
    \label{subfig:deps_quartz_full_a}
        \includegraphics[width=\perfsubfigwidth\textwidth]{figures/new-quartz/deps_quartz_total_fig.pdf}
    }\hfill%
    \subfloat[][CDF of solve time for different {\tt clingo} configurations.]{
    \label{subfig:cdf_quartz_full_a}
        \includegraphics[width=\perfsubfigwidth\textwidth]{figures/new-quartz/cdf_quartz_total_fig.pdf}
    }\\
    \subfloat[][CDF of setup times for different cache sizes across all E4S packages.]{
    \label{subfig:cdf_e4s_quartz_load_a}
        \includegraphics[width=\perfsubfigwidth\textwidth]{figures/new-quartz/cdf_e4s_cache_quartz_setup_fig.pdf}
    }\hfill%
    \subfloat[][CDF of solve times for different cache sizes across all E4S packages.]{
    \label{subfig:cdf_e4s_quartz_solve_a}
        \includegraphics[width=\perfsubfigwidth\textwidth]{figures/new-quartz/cdf_e4s_cache_quartz_solve_fig.pdf}
    }\hfill%
    \subfloat[][CDF of total time for different cache sizes across all E4S packages.]{
    \label{subfig:cdf_e4s_quartz_full_a}
        \includegraphics[width=\perfsubfigwidth\textwidth]{figures/new-quartz/cdf_e4s_cache_quartz_total_fig.pdf}
    }\hfill%
    \subfloat[][CDF of old concretizer times and \clingo{} total solve times.]{
    \label{subfig:cdf_old_vs_new_a}

    \includegraphics[width=\perfsubfigwidth\textwidth]{figures/new-quartz/cdf_quartz_old_vs_clingo.pdf}
    }

    \caption{Performance figures of solving times across all packages on Quartz.\vspace{-1em}}
    \label{fig:all_perf}

\end{figure*}



\section{Conclusions}
- TBD
A paragraph of text goes here. Lots of text. Plenty of interesting
text. Text text text text text text text text text text text text text
text text text text text text text text text text text text text text
text text text text text text text text text text text text text text
text text text text text text text.
More fascinating text. Features galore, plethora of promises.

%-------------------------------------------------------------------------------
\section*{Acknowledgments}
%-------------------------------------------------------------------------------
The USENIX latex style is old and very tired, which is why
there's no \textbackslash{}acks command for you to use when
acknowledging. Sorry.

%-------------------------------------------------------------------------------
\section*{Availability}
%-------------------------------------------------------------------------------

USENIX program committees give extra points to submissions that are
backed by artifacts that are publicly available. If you made your code
or data available, it's worth mentioning this fact in a dedicated
section.

%-------------------------------------------------------------------------------
% References
%-------------------------------------------------------------------------------
{
   \footnotesize
   \bibliographystyle{abbrv}
   \bibliography{\jobname}
}

%%%%%%%%%%%%%%%%%%%%%%%%%%%%%%%%%%%%%%%%%%%%%%%%%%%%%%%%%%%%%%%%%%%%%%%%%%%%%%%%
\end{document}
%%%%%%%%%%%%%%%%%%%%%%%%%%%%%%%%%%%%%%%%%%%%%%%%%%%%%%%%%%%%%%%%%%%%%%%%%%%%%%%%

%%  LocalWords:  endnotes includegraphics fread ptr nobj noindent
%%  LocalWords:  pdflatex acks
