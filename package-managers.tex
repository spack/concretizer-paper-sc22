%-------------------------------------------------------------------------------
\section{Package Managers}
%-------------------------------------------------------------------------------
\label{sec:package-managers}

The idea of ``sofware release management'' for integrated sets of packages dates back to
a 1997 paper from Ven der Hoek et al.~\cite{van1997software}, and the first Linux
package managers emerged around the same time~\cite{rpm,apt}. Since then, the number and
complexity of software packages has grown enormously, as have the use cases for package
managers.

\subsection{Single-prefix Package Managers}

Package managers such as RPM, Yum, and APT~\cite{foster+:rpm03,silva:apt01,yum} are used
to manage the system packages of most Linux distributions. These tools focus on managing
one software stack, built with one compiler, which works well for system software and
drivers that underly all software on the system. They are designed so that software can
be upgraded in place for bugfixes and security updates. All software goes in to a single
{\it prefix}, e.g., {\tt /usr}, and for this reason, only one version of any package can
be installed at a time. Upgrades are prioritized over reproducibility, and users do not
have great flexibility over precise package versions and configurations.

Even in this limited package/version configuration space, dependency solving is
NP-complete~\cite{dicosmo:edos,mancinelli+:ase06-foss-distros} because any two
non-overlapping version constraints can cause a conflict. The Common Upgradeability
Description Format (CUDF) attempts to standardize an interface for external package
solvers~\cite{abate2012dependency,abate-2013-modular-package-manager} and has enabled
many encodings of the single-prefix upgrade problem: Mixed-Integer Linear Programming,
Boolean Optimization, and Answer Set
Programming~\cite{tucker+:icse07-opium,michel+:lococo2010,argelich+:lococo2010,gebser+:2011-aspcud}.
Most modern package solvers still use ad-hoc implementations instead of a common
external solver~\cite{abate2020dependency}.

\subsection{Language Package Managers}

Most modern language implementations also include their own package
managers~\cite{npm,pip,cargo,weizenbaum:pubgrub18}. The solver requirements for language
package managers are different from those in a linux distribution; they typically allow
multiple versions of the same package to be installed to support developer workflows and
reproducibility. Conflicts can arise, as most langauge module systems do not support
{\it linking} multiple versions of one package into a single executable. Javascript is a
notable exception to this. Language package managers still solve only for package
names and versions, ignoring compilers, build options, targets and other options that
expand the solution space. Most language package managers implement their own ad-hoc
solvers. Historically, these solvers were not complete, as
implementing a performant SAT solver in all but the lowest-level languages is quite
difficult. However, there has been a recent trend towards complete
solvers~\cite{pip-new-resolver,weizenbaum:pubgrub18} as complexity of language
ecosystems has grown. These systems do not support optimization beyond selecting recent
versions~\cite{abate2020dependency}.

\subsection{Functional and HPC Package Managers}

So-called ``functional'' package managers like
Nix~\cite{dolstra+:icfp08,dolstra+:lisa04} and Guix~\cite{courtes-guix-2015} allow users
to install arbitrarily many configurations of a given package. Like many HPC
deployments, they install each configuration to a unique prefix, but instead of a
human-readable name, they ensure that each installation gets a unique prefix by using a
hash of the bits of the installation itself. These systems do not use solvers to resolve
dependencies. Rather, they have very little conditional logic and rely on the specific
package recipes in a repository. As mentioned, Easybuild~\cite{hoste+:pyhpc12} is
similar in this regard, as it allows multiple configurations of packages by duplicating
package recipes and adhering to a strict naming convention to differentiate different
installations. In all of these systems, users must modify code in package recipes to
change the way dependencies are resolved.
